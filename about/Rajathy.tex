\documentclass[12pt,twoside]{article}
\usepackage[charter]{mathdesign}
\usepackage{microtype}
\usepackage{setspace}
\usepackage[a4paper,margin=0.1in,heightrounded]{geometry}
\title{Demand Side management in smart Grid --- A Game Theory Approach}
\author{R. Rajathy}
\date{30.11.2016}
%
\pagestyle{empty}
\begin{document}
  \maketitle
  \thispagestyle{empty}
  \onehalfspacing
  The future smart grid is envisioned as a large-scale cyber-physical system encompassing advanced power, communications, control, and computing technologies. In order to accommodate these technologies, it will have to build on solid mathematical tools that can ensure an efficient and robust operation of such heterogeneous and large-scale cyber physical systems. Inherently, the smart grid is a power network composed of intelligent nodes that can operate, communicate, and interact, autonomously, in order to efficiently deliver power and electricity to their consumers. This heterogeneous nature of the smart grid motivates the adoption of advanced techniques for overcoming the various technical challenges at different levels such as design, control, and implementation.

As the challenges in the smart Grid are enormous, some seek more attention for welfare of the consumers. Demand side management is one such challenge to be focused which not only benefits the consumers, but also to the utility in many ways.

Smart grid technology is envisioned as future power systems with advanced metering infrastructure (AMI), energy storage systems (ESS), sensing technologies, demand response (DR) control methodologies, and communication technologies at transmission and distribution levels in order to optimize electricity resource usage in an intelligent fashion.

A key advantage of the SG is flexible, bi-directional demand side management (DSM) for residential smart grid. Benefiting from information exchange between supply and demand, DSM is able to reduce peak electricity loads and increase the reliability of the power grid. Utilities and system operators can effectively manage power generation and through incentives, encourage the shifting of high-energy demand household appliances to off-peak hours. Cost sensitive consumers are able to adjust their demand according to time differentiated electricity pricing and make appropriate consumption scheduling decisions. Recently, there have been several studies detailing DSM approaches in residential grid networks.

DSM programs encourage users to shift their usage of high-power appliances to off-peak hours by providing economic incentives to consumers. Among different techniques considered for DSM, voluntary load management, direct load control, dynamic pricing is known as an effective means to encourage users to consume wisely and more efficiently. By reflecting the hourly changes in the wholesale electricity price to the demand side, users pay what the electricity is worth at different times of day and are consequently more willing to reduce their load at peak hours.
Though, there are various methods to find solutions for demand side management, Game theory become a prominent tool in the design and analysis of smart grids. There is a need to deploy novel models and algorithms that can capture the following characteristics of the emerging smart grid: (i)- the need for distributed operation of the smart grid nodes for communication and control purposes, (ii)- the heterogeneous nature of the smart grid which is typically composed of a variety of nodes such as micro-grids, smart meters, appliances, and others, each of which having different capabilities and objectives, (iii)- the need for efficiently integrating advanced
techniques from power systems, communications, and signal processing, and (iv)- the need for low-complexity distributed algorithms that can efficiently represent competitive or collaborative scenarios between the various entities of the smart grid. In this context, game theory could constitute a robust framework that can address many of these challenges.

Game theory is a mathematical framework that can be divided into two main branches: non-cooperative game theory and cooperative game theory. Non-cooperative game theory can be used to analyze the strategic decision making processes of a number of independent entities, i.e., players, that have partially or totally conflicting interests over the outcome of a decision process which is affected by their actions. Essentially, non-cooperative games can be seen as capturing a distributed decision making process that allows the players to optimize, without any coordination or communication, objective functions coupled in the actions of the involved players. Non-cooperative does not always imply that the players do not cooperate, but it means that, any cooperation that a rises must be self-enforcing with no communication or coordination of strategic choices among the players.

Within the context of smart grids, the applications of non-cooperative games and of learning algorithms are numerous. On the one hand, non-cooperative games can be used to perform distributed demand-side management and real-time monitoring or to deploy and control micro-grids. On the other hand, economical factors such as markets and dynamic pricing are an essential part of the smart grid. In this respect, non-cooperative games provide several frameworks ranging from classical non-cooperative Nash games to advanced dynamic games which enable to optimize and devise pricing strategies that adapt to the nature of the grid.

In smart grids, with the deployment of advanced networking technologies, it is often possible to enable a limited form of communication between the nodes which paves the way for introducing cooperative game-theoretic approaches. In fact, the integration of power, communication, and networking technologies in future grids opens up the door for several prospective applications in which smart grid nodes can cooperate so as to improve the robustness and efficiency of the grid. One simple example would be to apply cooperative game theory in order to study how relaying can be performed in a large-scale smart grid network so as to improve the efficiency of the communication links between
Smart grid elements.

Clearly, game-theoretic approaches present a promising tool for the analysis of smart grid systems. Nonetheless, the advantages of applying distributed game-theoretic techniques in any complex system such as the smart grid are accompanied by key technical challenges. First, one of the underlying assumptions in classical game-theoretic designs is that the players are rational, i.e., each player makes its strategy choice so as to optimize its individual utility and, thus, confirm to some notion of equilibrium play. In practical control systems such as the smart grid, as the individual nodes of the system interact and learn their strategies, one or more nodes might deviate from the intended play and make non-rational decisions, i.e., choose unintended strategies, due to various factors such a failure or delay in learning.

These inaccurate strategy choices can eventually lead to a non-convergence to the desired equilibrium and, hence, impact the overall control system stability. The impact of such bad decisions becomes more severe in practical deployments in which the smallest perturbation to the system stability can lead to outages or other detrimental consequences. As a result, when designing game-theoretic models for the smart grid, it is imperative to emphasize robustness in the model and algorithm design.
So the application of game theory to demand side management plays a vital role compared to any other analytical tools found in the literature.

\end{document} 