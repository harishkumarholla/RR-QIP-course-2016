\documentclass[12pt]{article}
\usepackage[charter]{mathdesign}
\usepackage{microtype}
\usepackage{setspace}
\usepackage[a4paper,margin=1in,heightrounded]{geometry}
\title{About the Course}
\author{R. Rajathy}
%
\begin{document}
  \maketitle
  \doublespacing
  Respected principal Prof.\,P. Dananjayan, Respected chief guest Prof.\,S. Arul Daniel, HOD of EEE Prof.\,Alamelu Nachiappan, HOD of IT Prof.\, S. Saraswady, Dear colleagues and participants, I wish a very good morning to you all. I am indeed happy to stand before you to speak about this AICTE QIP sponsored one week short term course on \emph{ICT Solutions for Issues and Challenges in Smart Grid Technology}.

 With the increasing demands for stable, reliable and un-interrupted power supply, utilities all over the world are adopting Smart Grid technologies. The mantra \emph{Smart Grid}  not only gives opportunities and challenges to Electrical Engineers, but also opens a wide gateway to Communication Engineers, Instrumentation Engineers and of course Software Engineers as well. The issues and challenges in this modern technology are enormous and at least for another 30 years this technology will keep us engaged in the process of research.  However, the success of this technology depends on smart and knowledgeable engineers like us, who operate the system.

At the outset, We, the coordinators, thank our principal Prof.\,P. Danajayan,  Dean (Research) and QIP coordinator Prof.\,S. Himavathy  and Head of Department EEE, Prof.\,Alamelu Nachaiappan and Head of the Department IT,  Prof.\,S. Saraswathy for the opportunity  given to us to coordinate the first interdisciplinary short term course in our college.

We have sent around 600 information brochures across the country through conventional post. The information was also posted in our website. We received a good number of 200 applications through online registration. It was really a great challenge to select the participants as the number is limited to 35. We have given equal importance to all relevant disciplines of engineering and short listed according to their area of interest.

Lot of effort has been taken in preparing the course schedule in a way that it will cover all research issues related to SG Technology. Every day we will have 4 sessions --- two in the forenoon and two in the afternoon. On the first day we have lectures on \emph{Hybrid Vehicles and Hybrid Power plants} by our chief guest Prof.\,S. Arul Daniel followed by Prof.\,R. Gnanadass and Prof.\,Santhi Baskaran lecturing on \emph{Smart Distribution system and smart communities} and \emph{Introduction of Cloud computing} respectively.

On the second day, the first session will be about \emph{Application of Cloud computing to Power systems} by Dr.\,V. Gomathi from anna university and Dr.\,A. Kavitha from the same institution will talk about \emph{Integration of Renewable Energy resources} in the second session.  Afternoon sessions will be engaged by Prof.\,S. Jeevananthan and Dr.\,R. Sundaramurthy who will enlighten us with \emph{Power Electronics in Smart Grid} and \emph{Internet of things -- A Designers perspective} respectively.

The third day will start with \emph{Demand side management -- A Game Theory approach} by Dr.\,R. Rajathy (i.e., me) and followed by \emph{Multi-objective Optimization techniques for Smart Grid applications} by Dr.\,K. Saruladha. In the afternoon sessions we will have an industrial visit to Puducherry electricity department and smart grid pilot project, Puducherry where Superindent Engineer Er.\, Ravi and Assistant Engineer Er.\, Vivek Antony will address the participants on the \emph{Implementation of smart grid pilot project in Puducherry}.

On Thursday, Er.\,D. Karthiganesh from Dell International Services, Bangalore, will deliver a lecture on \emph{Big data analytics and Clustering Techniques  for smart grid applications}, followed by Prof.\,L. Nithyanadhan and Dr.\,G. Santhi talking about \emph{Internet of Power Things} and \emph{Communication and Networking Technolgies for Smart Grid applications} respectively.

On the final day, Dr.\,R. Rajeswari from GCT, Coimbatore will deliver a lecture on \emph{Synchro-phasor Technology} followed by Dr.\,V. Geetha speaking on \emph{IT related issues and Challenges in Smart Grid Technology}.

We presume that we all would have been stressed out by this time. To relieve this stress, we have arranged a session in the afternoon by Dr.\,Jayestri Kurushev from MTPG \& RIHS. She will talk about \emph{A to Z of Stress Management} and teach us ways to withstand and relieve stress. The final session is reserved for Group Discussion and feedback. We will have project presentations, wherein the participants will present a new proposal in group, based on the ideas they received through this  course.  At the end of this session we shall have the valedictory function with distribution of prizes for project presentation, best summary writing etc and certificates for the successful participation.

Last but not the least, the successful completion of this course depends on the active participation of participants. We, the co-ordinators rely on your cooperation in this regard. I request the participants to attend all the classes, interact with the speakers and make the sessions lively.

\begin{center}
  \Large Wish you a happy and memorable week ahead!

  Thank you
\end{center}
\end{document} 